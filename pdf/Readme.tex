\PassOptionsToPackage{unicode=true}{hyperref} % options for packages loaded elsewhere
\PassOptionsToPackage{hyphens}{url}
%
\documentclass[]{article}
\usepackage{lmodern}
\usepackage{amssymb,amsmath}
\usepackage{ifxetex,ifluatex}
\usepackage{fixltx2e} % provides \textsubscript
\ifnum 0\ifxetex 1\fi\ifluatex 1\fi=0 % if pdftex
  \usepackage[T1]{fontenc}
  \usepackage[utf8]{inputenc}
  \usepackage{textcomp} % provides euro and other symbols
\else % if luatex or xelatex
  \usepackage{unicode-math}
  \defaultfontfeatures{Ligatures=TeX,Scale=MatchLowercase}
\fi
% use upquote if available, for straight quotes in verbatim environments
\IfFileExists{upquote.sty}{\usepackage{upquote}}{}
% use microtype if available
\IfFileExists{microtype.sty}{%
\usepackage[]{microtype}
\UseMicrotypeSet[protrusion]{basicmath} % disable protrusion for tt fonts
}{}
\IfFileExists{parskip.sty}{%
\usepackage{parskip}
}{% else
\setlength{\parindent}{0pt}
\setlength{\parskip}{6pt plus 2pt minus 1pt}
}
\usepackage{hyperref}
\hypersetup{
            pdftitle={Readme},
            pdfauthor={Jorge Santos Neila \& Javier Cela Lopez},
            pdfborder={0 0 0},
            breaklinks=true}
\urlstyle{same}  % don't use monospace font for urls
\usepackage[margin=1in]{geometry}
\usepackage{graphicx,grffile}
\makeatletter
\def\maxwidth{\ifdim\Gin@nat@width>\linewidth\linewidth\else\Gin@nat@width\fi}
\def\maxheight{\ifdim\Gin@nat@height>\textheight\textheight\else\Gin@nat@height\fi}
\makeatother
% Scale images if necessary, so that they will not overflow the page
% margins by default, and it is still possible to overwrite the defaults
% using explicit options in \includegraphics[width, height, ...]{}
\setkeys{Gin}{width=\maxwidth,height=\maxheight,keepaspectratio}
\setlength{\emergencystretch}{3em}  % prevent overfull lines
\providecommand{\tightlist}{%
  \setlength{\itemsep}{0pt}\setlength{\parskip}{0pt}}
\setcounter{secnumdepth}{0}
% Redefines (sub)paragraphs to behave more like sections
\ifx\paragraph\undefined\else
\let\oldparagraph\paragraph
\renewcommand{\paragraph}[1]{\oldparagraph{#1}\mbox{}}
\fi
\ifx\subparagraph\undefined\else
\let\oldsubparagraph\subparagraph
\renewcommand{\subparagraph}[1]{\oldsubparagraph{#1}\mbox{}}
\fi

% set default figure placement to htbp
\makeatletter
\def\fps@figure{htbp}
\makeatother


\title{Readme}
\author{Jorge Santos Neila \& Javier Cela Lopez}
\date{11 de abril de 2020}

\begin{document}
\maketitle

{
\setcounter{tocdepth}{2}
\tableofcontents
}
\begin{center}\rule{0.5\linewidth}{0.5pt}\end{center}

\hypertarget{descripciuxf3n}{%
\section{Descripción}\label{descripciuxf3n}}

El conjunto de datos generado como parte de esta actividad práctica
reúne diferentes características de accidentes de avión ocurridos a
nivel mundial entre los años 1920 y 2017. Algunas de las variables que
se recogen en el conjunto de datos son la fecha, la hora, la
localización, el número de pasajeros o el de fallecidos.

\begin{center}\rule{0.5\linewidth}{0.5pt}\end{center}

\hypertarget{imagen-identificativa}{%
\section{Imagen identificativa}\label{imagen-identificativa}}

\begin{center}\rule{0.5\linewidth}{0.5pt}\end{center}

\hypertarget{contexto}{%
\section{Contexto}\label{contexto}}

Como se ha comentado, la materia del conjunto de datos se corresponde
con accidentes de avión que han tenido lugar durante el último siglo en
todo el mundo. Entre ellos pueden encontrarse accidentes de aviación de
todo tipo: desde helicópteros militares o de transporte hasta aviones
comerciales de pasajeros, pasando por todos aquellos que efectuaban
maniobras de entrenamiento.

\begin{center}\rule{0.5\linewidth}{0.5pt}\end{center}

\hypertarget{contenido}{%
\section{Contenido}\label{contenido}}

Para cada accidente, el cual se corresponde con un registro en el
conjunto de datos, se recogen las siguientes características:

\begin{itemize}
\tightlist
\item
  \textbf{Date}: día en el que tuvo lugar el accidente en el formato
  dd/mm/aaa.\\
\item
  \textbf{Time}: hora del accidente en formato de 24 horas.\\
\item
  \textbf{Location}: lugar del accidente.\\
\item
  \textbf{Latitude}: coordenada geográfica correspondiente a la latitud
  obtenida a partir de la localización y la librería \emph{Geopy}.\\
\item
  \textbf{Longitude}: coordenada geográfica correspondiente a la
  longitud obtenida a partir de la localización y la librería
  \emph{Geopy}.\\
\item
  \textbf{Operator}: nombre de la aerolínea o compañía de vuelos
  propietaria del avión o aviones involucrados en el accidente.\\
\item
  \textbf{Flight\#}: número del vuelo.\\
\item
  \textbf{Route}: ruta prevista por los aviones implicados.\\
\item
  \textbf{ACType}: tipo de avión.\\
\item
  \textbf{Registration}: registro ICAO (International Civil Aviation
  Organization) del avión.\\
\item
  \textbf{cn/ln}: construcción o número de serie/línea o número de
  fuselaje.\\
\item
  \textbf{Aboard}: número total de personas a bordo (pasajeros +
  tripulación).\\
\item
  \textbf{Fatalities}: número de víctimas mortales.\\
\item
  \textbf{Ground}: número de fallecidos en tierra.\\
\item
  \textbf{Summary}: resumen del accidente.\\
\item
  \textbf{Reason}: causa del accidente. Constituye una categoría que se
  ha asignado en base al resumen aplicando técnicas de minería de textos
  (\emph{Text Mining}).
\end{itemize}

Los autores de la web \emph{PlaneCrashInfo} llevan recopilando
información sobre estos accidentes desde el año 1997. Como fuentes de
información utilizadas (listadas en este enlace:
\url{http://www.planecrashinfo.com/reference.htm}) destacan periódicos,
revistas y libros, entre otros.

\begin{center}\rule{0.5\linewidth}{0.5pt}\end{center}

\hypertarget{agradecimientos}{%
\section{Agradecimientos}\label{agradecimientos}}

Los datos han sido recolectados desde la base de datos online
\href{http://www.planecrashinfo.com/database.htm}{PlaneCrashInfo}. Para
ello, se ha hecho uso del lenguaje de programación Python y de técnicas
de \emph{Web Scraping} para extraer la información alojada en las
páginas HTML.

\begin{center}\rule{0.5\linewidth}{0.5pt}\end{center}

\hypertarget{inspiraciuxf3n}{%
\section{Inspiración}\label{inspiraciuxf3n}}

El presente conjunto de datos podría utilizarse en ámbitos muy diversos.
Uno de ellos podría ser en el periodístico, en el que disponer de los
datos de la gran mayoría de accidentes de avión acontecidos en la
historia podría valer para sacar a relucir aquellos que puedan resultar
más interesante de cara a la realización de un reportaje.

También podría ser de gran utilidad en el campo de la \emph{minería de
datos}, a la hora de elaborar modelos predictivos (como por ejemplo
árboles de decisión o redes neuronales). Así, se podría querer elaborar
un modelo que permita predecir el número de víctimas mortales que podría
ocasionar un avión en caso de accidente dadas las características del
vuelo.

\begin{center}\rule{0.5\linewidth}{0.5pt}\end{center}

\hypertarget{licencia}{%
\section{Licencia}\label{licencia}}

La licencia escogida para la publicación de este conjunto de datos ha
sido \textbf{CC BY-SA 4.0 License}. Los motivos que han llevado a la
elección de esta licencia tienen que ver con la idoneidad de las
cláusulas que esta presenta en relación con el trabajo realizado:

\begin{itemize}
\item
  \emph{Se debe proveer el nombre del creador del conjunto de datos
  generado, indicando los cambios que se han realizado}. De esta manera,
  se reconoce el trabajo ajeno y en qué medida se han realizado
  aportaciones en relación con el trabajo original.
\item
  \emph{Se permite un uso comercial}. Esto haría que incrementen las
  probabilidades de que una empresa utilice los datos generados y
  realicen trabajos de calidad que reporten cierto reconocimiento al
  autor original.
\item
  \emph{Las contribuciones realizadas a posteriori sobre el trabajo
  publicado bajo esta licencia deberán distribuirse bajo la misma}. Esto
  hace que el trabajo del autor original continúe distribuyéndose bajo
  los términos que él mismo planteó.
\end{itemize}

\begin{center}\rule{0.5\linewidth}{0.5pt}\end{center}

\hypertarget{cuxf3digo-fuente-y-dataset}{%
\section{Código fuente y dataset}\label{cuxf3digo-fuente-y-dataset}}

Tanto el código fuente escrito para la extracción de datos como el
dataset generado pueden ser accedidos a través de
\href{https://github.com/tteguayco/Web-scraping}{este enlace}.

\begin{center}\rule{0.5\linewidth}{0.5pt}\end{center}

\hypertarget{recursos}{%
\section{Recursos}\label{recursos}}

\begin{enumerate}
\def\labelenumi{\arabic{enumi}.}
\tightlist
\item
  Lawson, R. (2015). Web Scraping with Python. Packt Publishing
  Ltd.~Chapter 2. Scraping the Data\\
\item
  Mitchel, R. (2015). Web Scraping with Python: Collecting Data from the
  Modern Web. O'Reilly Media, Inc.~Chapter 1. Your First Web Scraper.
\end{enumerate}

\end{document}
